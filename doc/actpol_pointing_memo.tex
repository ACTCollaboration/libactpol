\documentclass[10pt,preprint]{aastex}
\usepackage{natbib}
\bibliographystyle{apj}

%\slugcomment{\today}

\newcommand{\Ovec}{{\vec O}}
\newcommand{\rhat}{{\hat r}}
\newcommand{\east}{{\hat E}}
\newcommand{\rvec}{{\vec r}}
\newcommand{\uvec}{{\hat r}}
\newcommand{\zhat}{{\hat z}}
\newcommand{\slalib}{\texttt{slaLib}}
\newcommand{\refraction}{\mathrm{slaRefro}}
\newcommand{\alt}{\mathrm{Alt}}
\newcommand{\az}{\mathrm{Az}}
\newcommand{\ITRS}{\mathrm{ITRS}}
\newcommand{\GCRS}{\mathrm{GCRS}}
\newcommand{\JD}{\mathrm{JD}}
\newcommand{\UTone}{\mathrm{UT1}}
\newcommand{\UTC}{\mathrm{UTC}}
\newcommand{\ENU}{\mathrm{ENU}}
\newcommand{\NWU}{\mathrm{NWU}}
\newcommand{\UEN}{\mathrm{UEN}}
\newcommand{\LA}{\mathrm{LA}}
\newcommand{\IERS}{\mathrm{IERS}}
\newcommand{\rtopo}{{\hat r}_\mathrm{topo}}

\begin{document}
\title{ACTpol Pointing}
\author{Mike Nolta\\\today}

\begin{abstract}
A short note on how to convert from ACTpol to celestial coordinates.
Effects smaller than $0.1\arcsec$ are neglected.
\end{abstract}

\section{Miasma Theory}
%\section{``Since the beginning of time, man has yearned to destroy the atmosphere.''}
%\subsection{Observed $\to$ Topocentric}

Atmospheric refraction makes objects appear higher in the sky than they really are,
and so we start by applying an altitude adjustment:
\begin{equation}
\alt' = \alt + \refraction(\alt,\{p_i\})
\end{equation}
where $\refraction$ is a function in the {\slalib} library, and $\{p_i\}$ is a set of parameters describing the site
conditions (elevation, pressure, temperature, \dots).
\begin{eqnarray}
\rtopo &=& \left(\begin{array}{c}
\sin \alt'\\
\cos\alt' \sin\az \\
\cos\alt' \cos\az \\
\end{array}\right)
\end{eqnarray}
Note that we've chosen a slightly unusual basis ordering: UEN, or ``up, east, north''.

\section{A Journey to the Center of the Earth}

Now we want to shift our coordinates to the Earth's barycenter, which requires
a combination of a boost (to cancel out the Earth's rotation), a rotation, and a translation.

ACTpol's position on the surface of the Earth, in geocentric coordinates, is
\begin{eqnarray}
\Ovec_\ITRS &=& \left(\begin{array}{l}
(aC+h_G) \cos\phi_G \cos\lambda_G \\
(aC+h_G) \cos\phi_G \sin\lambda_G \\
(aS+h_G) \sin\phi_G \\
\end{array}\right)
\end{eqnarray}
where
\begin{eqnarray}
1/C^2 &=& \cos^2\phi_G + (1-f)^2\sin^2\phi_G \\
S &=& (1-f)^2 C \\
f &=& 1/298.257223563 \\
a &=& 6378137\,\mathrm{m}
\end{eqnarray}
The values of $a$ and $f$ are from WGS84.
At our level of precision GPS $\approx$ WGS84 $\approx$ ITRS, so we'll use them interchangeably.

%\subsection{Topocentric $\to$ Local Apparent}

Because we observe in a moving reference frame, 
``Diurnal aberration'' refers to the aberration caused by the Earth's rotation.
 tilts observed positions towards the direction of motion.
For $v/c \ll 1$, the amplitude of the effect is $v\sin\theta/c$, where $\theta$ is the angle between
object and direction of motion.

ACTpol's velocity due to the Earth's rotation is
\begin{eqnarray}
\vec v/c &=& \frac{\omega}{c}\zhat_\ITRS\times\Ovec_\ITRS
\\
&=& \frac{\omega}{c} (aC+h_G) \cos\phi_G \east
\\
&\approx& 0.295043 \arcsec\east
\end{eqnarray}
where $\omega = 7.292115\times10^{-5}\,\mathrm{rad/s}$ is the Earth's nominal mean angular speed,
and $\east$ is unit vector pointing east.
Let $\vec n = \rtopo\times\east$.
Since $|\vec n| = \sin\theta$ where $\theta$ is the angle between $\rtopo$ and $\east$,
\begin{eqnarray}
\rhat_\LA = R_{\hat n}(-0.295043 \arcsec|\vec n|) \rtopo
\end{eqnarray}

%\section{Local Apparent $\to$ ITRS}

Now that we've corrected for refraction and diurnal aberration, we need to rotate into geocentric
coordinates.\footnote{Well, technically we need to translate first, and then rotate, but we'll assume the target is
infinitely far away and ignore the translation. So this may not be correct for planets.}
Because we chose the UEN basis for $\rhat_\LA$,
\begin{eqnarray}
%\rhat_\ITRS &=& R_3(-\lambda_G) R_2(\phi_G) \rhat_\LA
%\\
\rhat_\ITRS &=& R_3(-\lambda_G) R_2(\phi_G-\pi/2) R_3(\pi) \rhat_\NWU
%\\
%\rhat_\ITRS &=& R_3(-\lambda_G) R_2(\phi_G-\pi/2) R_3(-\pi/2) \rhat_\ENU
\end{eqnarray}

\section{ITRS $\to$ GCRS}

Fig.~5.1 of \cite{iers2010}

\begin{eqnarray}
\rhat_\GCRS &=& R_3(-E) R_2(-d) R_3(E + s - \theta - s')R_2(x_p)R_1(y_p) \rhat_\ITRS \\
\theta &=& 2\pi(0.7790572732640 + 1.00273781191135448 D_U)
\end{eqnarray}
where
\begin{eqnarray}
(X,Y,Z) &=& (\sin d \cos E, \sin d \sin E, \cos d)
\end{eqnarray}
\begin{enumerate}
\item $(X,Y,s)$ is calculated by \texttt{iauXys06a}
\item \texttt{iauEra00} for $\theta$
\item $s'\approx0$
\item $(\Delta X,\Delta Y)_{2006} \approx (0,0)$
\item $(x_p,y_p) \approx (x,y)_\IERS$
\end{enumerate}
%$D_U=\JD(\UTone) - 2451545.0$, days since 2000 Jan 1, 12h UT1\\

\section{Plunge Orbit}
%\section{GCRS $\to$ BCRS}

Shift from the Earth's barycenter to the Solar system's barycenter.
Annual aberration.

\bibliography{references}
\appendix

\section{Glossary}

\begin{tabular}{ll}
BCRS & Barycentric Celestial Reference System \\
CIO & Celestial Intermediate Origin \\
CIP & Celestial Intermediate Pole \\
GCRS & Geocentric Celestial Reference System \\
GPS & Global Positioning System \\
ICRS & International Celestial Reference System \\
ITRS & International Terrestrial Reference System \\
JD & Julian date \\
UT1 & Universal Time \\
UTC & Coordinated Universal Time \\
WGS84 & World Geodetic System 1984 \\
$\az$ & observed azimuth \\
$\alt$ & observed elevation \\
$h_G$ & geodetic height = 5188 m \\
$\lambda_G$ & geodetic east longitude = $-67.7876^\circ$ \\
$\phi_G$ & geodetic latitude = $-22.9585^\circ$ \\
$R_1(a)$ & $x$-rotation matrix \\
$R_2(a)$ & $y$-rotation matrix \\
$R_3(a)$ & $z$-rotation matrix \\
$s$ & CIO locator \\
$\theta$ & Earth Rotation Angle \\
$x_p$, $y_p$ & standard polar wobble parameters \\
$X$, $Y$, $Z$ & components of unit vector towards the CIP in the GCRS
\end{tabular}

\begin{eqnarray}
R_1(\alpha) &=&
\left(\begin{array}{ccc}
1 & 0 & 0 \\
0 & \cos\alpha & \sin\alpha \\
0 & -\sin\alpha & \cos\alpha \\
\end{array}\right)
\\
R_2(\alpha) &=&
\left(\begin{array}{ccc}
\cos\alpha & 0 & -\sin\alpha \\
0 & 1 & 0 \\
\sin\alpha & 0 & \cos\alpha \\
\end{array}\right)
\\
R_3(\alpha) &=&
\left(\begin{array}{ccc}
\cos\alpha & \sin\alpha & 0 \\
-\sin\alpha & \cos\alpha & 0 \\
0 & 0 & 1 \\
\end{array}\right)
\end{eqnarray}

\section{Example}

\end{document}
