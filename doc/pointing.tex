\documentclass[10pt,preprint]{aastex}
\usepackage{natbib}
\usepackage[normalem]{ulem}
\bibliographystyle{apj}

\newcommand{\Ovec}{{\vec O}}
\newcommand{\east}{{\hat E}}
\newcommand{\rvec}{{\vec r}}
\newcommand{\uvec}{{\hat r}}
\newcommand{\zhat}{{\hat z}}
\newcommand{\alt}{\mathrm{Alt}_0}
\newcommand{\az}{\mathrm{Az}_0}
\newcommand{\ITRS}{\mathrm{ITRS}}
\newcommand{\GCRS}{\mathrm{GCRS}}
\newcommand{\JD}{\mathrm{JD}}
\newcommand{\UTone}{\mathrm{UT1}}
\newcommand{\UTC}{\mathrm{UTC}}
\newcommand{\ENU}{\mathrm{ENU}}
\newcommand{\NWU}{\mathrm{NWU}}
\newcommand{\UEN}{\mathrm{UEN}}
\newcommand{\LA}{\mathrm{LA}}
\newcommand{\IERS}{\mathrm{IERS}}

\newcommand{\rhat}{{\hat r}}
\newcommand{\rhorn}{{\hat r}_\mathrm{horn}}
\newcommand{\rfocal}{{\hat r}_\mathrm{focal}}
\newcommand{\robs}{{\hat r}_\mathrm{obs}}
\newcommand{\rtopo}{{\hat r}_\mathrm{topo}}
\newcommand{\rgcrs}{{\hat r}_\mathrm{GCRS}}
\newcommand{\rbcrs}{{\hat r}_\mathrm{BCRS}}

\begin{document}
\title{ACTpol Pointing}
\author{Mike Nolta\\\today}

\begin{abstract}
How \texttt{libactpol} converts from telescope to celestial coordinates.
See the appendix for a glossary of definitions, and see \citet{usno179} for more
information about the coordinate systems.
\end{abstract}

\section{Overview}

\begin{enumerate}
\item Convert telescope encoders to horizon coordinates.
\item Correct for atmospheric refraction.
\item Boost to cancel Earth's rotation (diurnal abberation).
\item Translate to Earth barycenter (diurnal parallax) [currently neglected].
\item Rotate to crust-fixed coordinates (the ITRS).
\item Rotate to Earth-centered celestial coordinates (the GCRS).
\item Boost to cancel Earth's orbital velocity (annual aberration).
\item Translate to Solar barycenter (the BCRS/ICRS) (annual parallax) [currently neglected].
\end{enumerate}

\section{Step 1: Telescope $\to$ Horizon}

We start by defining feedhorn-centric coordinates:
$\hat x$ and $\hat y$ along the detector polarization directions
and $\hat z$ pointing toward the sky along the line of sight.
Next, we need to define the rotation from feedhorn- to boresight-centric coordinates:
\begin{eqnarray}
\rfocal = R_1(f_y)R_2(-f_x)R_3(f_\phi) \rhorn
\end{eqnarray}
This convention was chosen so that $f_x,f_y,f_\phi$ match Ben's array layout spreadsheet.
Fig.~\ref{fig:focalplane} shows the expected PA1 focalplane layout.
Adopting NWU (``north, west, up'') as the axes of our horizon coordinates,
\begin{eqnarray}
\robs = R_3(\az)R_2(\alt-\pi/2)R_3(-\pi) \rfocal
\end{eqnarray}
where $\alt,\az$ are the boresight coordinates.

\begin{figure}
\plotone{focalplane.eps}
\caption{The ACTpol PA1 focalplane. The $x$-axis is $(\rfocal\cdot\hat x)(180/\pi)$,
and similarly for the $y$-axis.\label{fig:focalplane}}
\end{figure}

\section{Step 2: Miasma Theory}

Atmospheric refraction makes objects appear higher in the sky than they really are,
so we need to apply an \sout{attitude} altitude adjustment
\begin{eqnarray}
\rtopo = R_3(\az)R_2(-\xi_0)R_3(-\az) \robs.
\end{eqnarray}
Note that, for the moment, we have neglected the variation of $\xi$ across the array,
which is ${\cal O}(1\arcsec)$.

To calculate $\xi$, we use the \texttt{slalib:slaRefro} routine, which is
a function of the telescope position (elevation \& latitude), the observing
frequency, and the local weather conditions (pressure, temperature, relative humidity,
\& tropospheric lapse rate).
We fix the tropospheric lapse rate to its standard value, $6.5\,\mathrm{K/km}$.
Fig.~\ref{fig:ref} plots $\xi$ versus altitude for various weather conditions,
showing that $\xi$ is most sensitive to the temperature and humidity.

\begin{figure}
\plotone{refraction.eps}
\caption{ACTpol site refraction correction for various weather conditions.\label{fig:ref}}
\end{figure}

\section{Steps 3--5: Journey to the Center of the Earth}

Now we want to shift our coordinates to the Earth's barycenter, which requires
a combination of a boost (to cancel out the Earth's rotation), a rotation, and a translation.

Let $\vec v_r$ be ACTpol's velocity relative to the Earth's barycenter, and $\vec\beta_r\equiv\vec v_r/c$.
Since $|\vec\beta_r| \ll 1$, we can approximate the aberration correction as
\begin{eqnarray}
\rhat_\LA = R(\rtopo\times\vec\beta_r) \rtopo
\end{eqnarray}
where $R(\vec\delta)$ is a clockwise rotation about the $\hat\delta$-axis by $|\vec\delta|$ radians.

ACTpol's position on the surface of the Earth, in geocentric coordinates, is
\begin{eqnarray}
\Ovec_\ITRS &=& \left(\begin{array}{l}
(aC+h_G) \cos\phi_G \cos\lambda_G \\
(aC+h_G) \cos\phi_G \sin\lambda_G \\
(aS+h_G) \sin\phi_G \\
\end{array}\right)
\end{eqnarray}
where
\begin{eqnarray}
1/C^2 &=& \cos^2\phi_G + (1-f)^2\sin^2\phi_G \\
S &=& (1-f)^2 C \\
f &=& 1/298.257223563 \\
a &=& 6378137\,\mathrm{m}
\end{eqnarray}
The values of $a$ and $f$ are from WGS84.\footnote{
At our level of precision GPS $\approx$ WGS84 $\approx$ ITRS, so we'll use them interchangeably.}
Thus ACTpol's velocity due to the Earth's rotation is
\begin{eqnarray}
\vec\beta_r &=& \frac{\omega}{c}\zhat_\ITRS\times\Ovec_\ITRS
\\
&=& \frac{\omega}{c} (aC+h_G) \cos\phi_G \east
\\
&\approx& 0.295043 \arcsec\east
\end{eqnarray}
where $\omega = 7.292115\times10^{-5}\,\mathrm{rad/s}$ is the Earth's nominal mean angular speed,
and $\east$ is the unit vector pointing east.

Now that we've corrected for diurnal aberration, we need to translate and rotate into geocentric
coordinates. We'll assume the target is infinitely far away and ignore the translation.
Since we chose the NWU horizon basis,
\begin{eqnarray}
\rhat_\ITRS &=& R_3(-\lambda_G) R_2(\phi_G-\pi/2) R_3(-\pi) \rhat_\LA
\end{eqnarray}

\section{Step 6: ITRS $\to$ GCRS}

For our purposes the conversion from ITRS to GCRS is a black box, simply a complex time-dependent rotation matrix.
From Fig.~5.1 of \cite{iers2010}, we find
\begin{eqnarray}
\rhat_\GCRS &=& R_3(-E) R_2(-d) R_3(E + s - \theta - s')R_2(x_p)R_1(y_p) \rhat_\ITRS
\end{eqnarray}
where
\begin{enumerate}
\item $(\sin d \cos E, \sin d \sin E, \cos d) \equiv (X,Y,Z)
= (X,Y,\sqrt{1 - X^2 - Y^2})$.
\item $(X,Y,s)$ are calculated by \texttt{sofa:iauXys06a}.
We ignore the corrections $(\Delta X,\Delta Y)_{2006} \approx 0.1\,\mathrm{mas}$.
\item $\theta$ is calculated by \texttt{sofa:iauEra00}.
\item We ignore $s' = -47T\,\mu\mathrm{as}$, where $T$ is time in centuries since J2000.0.
\item $(x_p,y_p) \approx (x,y)_\IERS$ tabulated in IERS Bulletin A.
\end{enumerate}

\section{Steps 7--8: Stop the World, I Want to Get Off}

Let $\vec v_o$ be the velocity of Earth's barycenter relative to the solar barycenter, and $\vec\beta_o\equiv\vec v_o/c$.
\begin{eqnarray}
\rbcrs = R(\rgcrs\times\vec\beta_o) \rgcrs
\end{eqnarray}
We use the \texttt{sofa:iauEpv00} routine to calculate $\vec v_o$.
For efficiency, we only calculate $\vec v_o$ once every 2.4~seconds, which is approximately the time
it takes $\vec v_o$ to rotate by $0.1\arcsec$.

\nocite{*}
\bibliography{references}
\appendix

\section{Glossary}

\begin{tabular}{ll}
BCRS & Barycentric Celestial Reference System \\
CIO & Celestial Intermediate Origin \\
CIP & Celestial Intermediate Pole \\
GCRS & Geocentric Celestial Reference System \\
GPS & Global Positioning System \\
ICRS & International Celestial Reference System \\
IERS & International Earth Rotation \& Reference Systems Service \\
ITRS & International Terrestrial Reference System \\
JD & Julian date \\
UT1 & Universal Time \\
UTC & Coordinated Universal Time \\
WGS84 & World Geodetic System 1984 \\
$\az$ & observed boresight azimuth \\
$\alt$ & observed boresight elevation \\
$h_G$ & geodetic height (ACTpol = 5188 m) \\
$\lambda_G$ & geodetic east longitude (ACTpol = $-67.7876^\circ$) \\
$\phi_G$ & geodetic latitude (ACTpol = $-22.9585^\circ$) \\
$R_1(\psi)$ & $\hat x$-axis clockwise rotation matrix \\
$R_2(\psi)$ & $\hat y$-axis clockwise rotation matrix \\
$R_3(\psi)$ & $\hat z$-axis clockwise rotation matrix \\
$R(\vec\delta)$ & $\hat\delta$-axis rotation matrix \\
$s$ & CIO locator \\
$\theta$ & Earth Rotation Angle \\
$x_p$, $y_p$ & standard polar wobble parameters \\
$X$, $Y$, $Z$ & components of unit vector towards the CIP in the GCRS
\end{tabular}

\begin{eqnarray}
R_1(\alpha) &=&
\left(\begin{array}{ccc}
1 & 0 & 0 \\
0 & \cos\alpha & \sin\alpha \\
0 & -\sin\alpha & \cos\alpha \\
\end{array}\right)
\\
R_2(\alpha) &=&
\left(\begin{array}{ccc}
\cos\alpha & 0 & -\sin\alpha \\
0 & 1 & 0 \\
\sin\alpha & 0 & \cos\alpha \\
\end{array}\right)
\\
R_3(\alpha) &=&
\left(\begin{array}{ccc}
\cos\alpha & \sin\alpha & 0 \\
-\sin\alpha & \cos\alpha & 0 \\
0 & 0 & 1 \\
\end{array}\right)
\end{eqnarray}

\section{Polarization Convention}

Given a line of sight vector $\hat r$ and polarization direction $\hat p$,
$(\sin\gamma,\cos\gamma) = (\hat p \cdot \hat w, \hat p \cdot \hat n)$, where
\begin{eqnarray}
\hat w &\propto& \hat r \times \hat z
\\
\hat n &=& \hat w \times \hat r
\end{eqnarray}

\end{document}
